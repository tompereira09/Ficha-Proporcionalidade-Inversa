\documentclass[12pt]{article}

\title{Ficha Lugares Geométricos}
\author{Tomás Pereira}

\begin{document}
\maketitle

1.\\\\
Yp = $\frac{3}{4}\cdot4+2=3+2=5$\\
P(4,5)\\
k = $4\cdot5=20$\\
a = k\\
a = $20$\\\\
R: O valor de $a$ é 20.\\

2.\\\\
Ya = $\frac{16}{4}=4$\\
A(4,4)\\
$f(x)=ax+b$\\
a = $\frac{Ya-Yb}{Xa-Xb}$\\
a = $\frac{4-0}{4-(-2)}=\frac{4}{6}=\frac{2}{3}$\\
$4=\frac{2}{3}\cdot4+b\equiv4=2.7+b\equiv-b=2.7-4\equiv b=-2.7+4\equiv b=1.3$\\
$f(x)=\frac{2}{3}a+1.3$\\\\
R: A função $f$ é definida por $f(x)=\frac{2}{3}a+1.3$.\\\\\\\\\\\\

3.\\\\
A(2,12)
Ya = $3\cdot2^2=3\cdot4=12$\\
k = $2\cdot12=24$\\
a = $24$\\\\
R: O valor de $a$ é 24.\\

4.\\\\
R: C.\\

5.\\\\
A(2, )\\
Ya = $4\cdot2=8$\\
A(2,8)\\
k = $2\cdot8=16$\\
$g(x)=\frac{16}{x}$\\
$g(2)=\frac{16}{2}\equiv g(2)=8$\\

6.\\\\
$f(x)=ax^2$\\
$g(x)=\frac{k}{x}$\\
k = $4\cdot3=12$\\
$g(x)=\frac{12}{x}$\\
P(2, )\\
Yp = $\frac{12}{2}=6$\\
P(2,6)\\
$y=ax^2$\\
$6=a2^2\equiv6=4a\equiv a=\frac{6}{4}\equiv a=\frac{3}{2}$\\\\
R: O valor de $a$ é $\frac{3}{2}$.\\\\\\\\

7.\\\\
A(3, )\\
Ya = $\frac{2}{3}\cdot3^2=\frac{2}{3}\cdot\frac{9}{1}=\frac{18}{3}=6$\\
A(3,6)\\
$g(x)=\frac{18}{x}$\\
k = $3\cdot6=18$\\
$2=\frac{18}{x}\equiv2x=18\equiv x=\frac{18}{2}\equiv x=9$\\
B(9,2)\\
c = 9\\\\
R: O valor de $c$ é 9.\\

8.\\\\
k = $4\cdot12=48$\\
$\frac{48}{(4+2)}=\frac{48}{6}=8$\\\\
R: Cada amigo contribuiu com 8 euros.\\

9.\\\\
k = $10\cdot9=90$\\
$\frac{90}{a}=15\equiv90=15a\equiv15a=90\equiv a=\frac{90}{15}\equiv a=6$\\\\
R: O valor de $a$ é 6.\\

10.\\\\
$f(x)=\frac{6}{x}$\\
P(2, )\\
Yp = $\frac{6}{2}=3$\\
P(2,3)\\
$g(x)=ax^2$\\
$3=a2^2\equiv3=4a\equiv4a=3\equiv a=\frac{3}{4}$\\\\
R: O valor de $a$ é $\frac{3}{4}$.\\

11.\\\\
A(4, )\\
Ya = $\frac{8}{4}=2$\\
A(4,2)\\
$f(x)=ax^2$\\
B(3,2)\\
$2=a3^2\equiv2=a9\equiv a=\frac{2}{9}$\\\\
R: O valor de $a$ é $\frac{2}{9}$.\\

12.\\\\
P(3, )\\
$Yp=\frac{4}{3}\cdot3\equiv Yp=4\cdot3\equiv Yp=12$\\
k = $3\cdot12=36$\\
$g(x)=\frac{36}{x}$\\\\
R: A função $g$ é definida por $g(x)=\frac{36}{x}$.\\





\end{document}
